\documentclass[12pt]{jsarticle}
\usepackage{amsmath,amssymb}
\pagestyle{empty}
\begin{document}
\setlength{\baselineskip}{22pt}
\begin{flushleft}
{\huge 1}
 座標平面上の点$A(0, \, 0)$,$B(0, \, 1)$,$C(1, \, 1)$,$D(1, \, 0)$を考える。
実数$0<t<1$に対して,線分$AB$,$BC$,$CD$を$t:(1-t)$に内分する点をそれぞれ$P_t$,$Q_t$,$R_t$とし,
線分$P_tQ_t$,$Q_tR_t$を$t:(1-t)$に内分する点をそれぞれ$S_t$,$T_t$とする。
さらに,線分$S_tT_t$を$t:(1-t)$に内分する点を$U_t$とする。
また,点$A$を$U_0$,点$D$を$U_1$とする。
\begin{description}
\item[(1)]点$U_t$の座標を求めよ。
\item[(2)]$t$が$0 \leqq t \leqq 1$の範囲を動くときに点$U_t$が描く曲線と,
線分$AD$で囲まれた部分の面積を求めよ。
\item[(3)]$a$を$0<a<1$を満たす実数とする。
$t$が$0 \leqq t \leqq a$の範囲を動くときに点$U_t$が描く曲線の長さを,
$a$の多項式の形で求めよ。
\end{description}
\end{flushleft}
\end{document}
