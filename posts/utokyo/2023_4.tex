\documentclass[12pt]{jsarticle}
\usepackage{amsmath,amssymb}
\pagestyle{empty}
\begin{document}
\setlength{\baselineskip}{22pt}
\begin{flushleft}
{\huge 4}
 座標空間内の4点$O(0, \, 0, \, 0)$,$A(2, \, 0, \, 0)$,$B(1, \, 1, \, 1)$,$C(1, \, 2, \, 3)$を考える。
\begin{description}
\item[(1)]$\overrightarrow{OP}\perp\overrightarrow{OA}$,$\overrightarrow{OP}\perp\overrightarrow{OB}$,
$\overrightarrow{OP}\cdot\overrightarrow{OC}=1$を満たす点$P$の座標を求めよ。
\item[(2)]点$P$から直線$AB$に垂線を下ろし,その垂線と直線$AB$の交点を$H$とする。
$\overrightarrow{OH}$を$\overrightarrow{OA}$と$\overrightarrow{OB}$を用いて表せ。
\item[(3)]点$Q$を$\displaystyle\overrightarrow{OQ}=\frac{3}{4}\overrightarrow{OA}+\overrightarrow{OP}$に
より定め,$Q$を中心とする半径$r$の球面$S$を考える。
$S$が三角形$OHB$と共有点を持つような$r$の範囲を求めよ。
ただし,三角形$OHB$は3点$O$,$H$,$B$を含む平面内にあり,周とその内部からなるものとする。
\end{description}
\end{flushleft}
\end{document}
